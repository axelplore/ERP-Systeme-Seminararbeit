% !TeX root = first.tex

\section{Kritische Bewertung/Ausblick auf SAP SucessFactors} 
\label{sec:bewertungundausblick}

SAP-SucessFactors ist Weiterentwicklung des herkömmlichen HCM-Moduls. Die neue neue Anwendung enthält alle Funktionen des klassischen HCM-Moduls und bietet zusätzlich die Möglichkeit cloudbasiert auf die SAP-SuccessFactors zuzugreifen. Dies hat den Vorteil das die Nutzung eines SAP-Systems nicht die Kosten für das Hosting des Systems mit sich bringen. Hierdurch können Personal- und Hardware-Kosten gespart werden. 

\subsection{Bewertung des HCM-Moduls}
Das HCM-Modul von SAP bietet, wie die anderen Module des SAP-ERP-Systemns einen besonders großen Umfang von Funktionen, wodurch nahezu jedes Unternehmen welches sich dazu entscheidet dieses System nutzen zu wollen, seine Abläufe und Werte/Ziele abbilden kann. Eine bestehende Personalverwaltung kann durch die Vielzahl der Schulungsunterlagen komfortabel auf die Umstellung vorbereitet werden. Es gibt jedoch auch eine Vielzahl von Nachteilen. Gartner, eine Platform für den Wissensaustausch von Unternhemen veröffentlicht Bewertungen für bekannte Business to Business (B2B) Anwendungen. Aus diesen Bewertungen können Rückschlüsse auf die Schwachstellen des HCM-Moduls gezogen werden. 

\subsection{Vorteile und Nachteile}
Die Vorteile des HCM-Moduls belaufen sich auf die bereits in dieser Arbeit genannten Funktionen. Diese bieten in Kombination eine effiziente ganzheitliche Lösung für die Personalverwaltung eines Unternehmens. Die Fähigkeit die Organisationsstruktur des Unternehmens abbilden zu können bietet auf lange Sicht einen besseren Überblick über das Unternehmen und hilf zusätzlich im Voraus personelle Engpässe zu entdecken. 

Da das Modul HCM als Produkt darauf ausgelegt ist möglichst allen potentiellen Kunden die Fähigkeit zu geben es für ihre Prozesse zu nutzen wird die Fähigkeit speziellere Prozesse abzubilden beschränkt. Sollte dies trotzdem von Nöten sein, muss das gekaufte System eigenständig angepasst werden. Die zieht mit sich das es nun zumindest ein Team geben muss, welches sich ausschließlich mit Anpassungen im SAP-System beschäftigt. Wenn Prozesse auch mit zusätzlichen Anpassungen nicht implementiert werden können, muss eventuell auf die Lösungen/Produkte eines anderen Herstellers zurückgegriffen werden. Dadurch kann schnell ein Fleckenteppich von Lösungen für Sonderfälle entstehen, welcher den Vorteil, welchen man durch eine allumfassende Anwendung bekommt, wieder verwirft. 

\subsection{Aussichten und Vergleich mit dem neuen SuccessFactors}
\subsubsection{Unterschiede}
Der größte Unterschied zwischen Success Factors und dem herkömmlichen HCM-Modul ist die Option für eine Cloud-Lösung. Die Vorhandene Fähigkeit zur Integration mit der Cloud (Hana Cloud Integration) sorgt für mehr Flexibilität beim Kunden. Die Nutzung der Cloud bringt ebenfalls den Unterschied mit sich das das Preismodell flexibel gehalten werden kann. Kunden haben die Möglichkeit nur für die tatsächlich von ihnen verwendeten Ressourcen zu bezahlen, anstelle von einem festen Preis, der auch dann anfällt wenn eventuell nicht alle Funktionen der Komponenten verwendet werden. 

\subsubsection{Vorteile/Fortschritte gegenüber HCM}
Diverse Vorteile wurden bereits zuvor in dieser Arbeit genannt. Für ein interessiertes Unternehmen ist es jedoch besonders interessant zu wissen wie die neue Success-Factors-Lösung dabei helfen kann auch komplexe und für die jeweilige Industrie ungewöhnliche Prozesse in dem System abbilden zu können. SAP-Succes-Factors kann, durch die Nutzung der Hana-Cloud leichter skaliert werden. Es gibt keine Notwendigkeit mehr ein eigenes Teams für die Wartung und Verwaltung von Hardware unter eigener Verantwortung zu beschäftigen. Hierdurch lässt sich Success-Factors auch anfänglich leichter integrieren. Hierdurch können dauerhaft Kosten gespart werden, wodurch ein besserer Return on Investment (ROI) erzielt werden. Desweiteren wurden die Module der HCM-Komponente in der Success-Factor-Komponente weiter überarbeitet und verbessert. Diese sind nun verschlankt und bieten eine noch effektivere Möglichkeit das Personal eines Unternehmens zu verwalten. Durch die neuen Komponenten, die bessere Skalierbarkeit, dem flexibleren Preismodell und der Fähigkeit Success-Factors mit vorhandenen OnPremise-Lösungen zu integrieren bietet Success-Factors für die meisten Unternhemen einen Grund zum upgrade. 