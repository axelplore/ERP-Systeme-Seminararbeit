% !TeX root = first.tex

\section{Aufbau der Komponenten HCM} 
\label{sec:aufbauderkomponenten}

Die SAP-Komponente HCM ist so aufgebaut das alle Prozesse im Bereich „Human Capital Management“ eines Unternehmens abgebildet werden können. Aufzuteilen ist die Struktur der Komponente in die Organisationsstruktur, die Stammdaten und die jeweiligen Prozesse welche durch diese Komponente abgebildet werden können. Im folgenden möchte ich, um den Aufbau der Komponente darstellen zu können, den Aufbau einer Organisationsstruktur in HCM beschreiben. Den Aufbau der verwendeten Stammdaten innerhalb des Moduls beschreiben und die Abbildbaren Prozesse erläutern.

\subsection{Organisationsstruktur}
Der Aufbau der Komponente muss, um effektiv verwendet werden zu können, den realen Aufbau des Unternehmens widerspiegeln. Der erste Schritt bei der Verwendung des Systems sollte daher das eintragen der Mitarbeiter-Daten in die Organisationsstruktur seien.1 Die Organisationsstruktur des HCM-Moduls wird dafür verwendet um den organisatorischen Aufbau eines Unternehmens abbilden zu können. Sie besteht im wesentlichen aus drei verschiedenen Strukturen, welche zusammen die die Position und Zuständigkeit eines Mitarbeiters beschreiben. 

\subsection{Unternehmensstruktur}
Die Unternehmensstruktur beschreibt den Aufbau der einzelnen Geschäftsbereiche eines Unternehmens, die unterschiedlichen Standorte und eventuelle Unterteilungen innerhalb. Die Geschäftsbereiche werden durch Mandanten abgebildet, welche untereinander keine Daten austauschen. Sie sind also voneinander unabhängige Einheiten. Die nächste Unterteilung besteht aus dem Buchungskreis. Über den Buchungskreis können z.B. rechtlich vorgeschriebene Finanzaufstellungen erstellt werden. Der Buchungskreis arbeitet mit der Buchhaltung eines Unternehmens, also z.B. der SAP „FI“ (Finanzwesen) Komponente festgelegt. Die zwei folgenden Strukturen, dem Personalbereich und dem Sub-personalbereichh müssen jeweils einem Buchungskreis angehören und sind besonders für die Personalverwaltung von Abteilungen und Teams von Interesse.\footnote{Vgl. \cite{SAPSE2024a}}

\subsection{Personalstruktur}
Im Human-Capital-Management ist es das Ziel das man die die menschlichen Ressourcen, namentlich jeder Mitarbeiter der an der Wertschöpfungskette des Unternehmens beteiligt ist, strukturieren und effektiv verwalten kann.\footnote{Vgl. \cite{GablerWirtschaftslexikon2018}} In der Komponente HCM wird die Personalstruktur als Mitarbeitergruppen/-kreis und deren Abrechnungsbereichen abgebildet. Gruppen können aus Externen und Internen Mitarbeitern oder aus Aktiven-  und Pensionierten Mitarbeitern aufgebaut sein. Die Kreise werden dafür verwendet um einen groben Überblick über den aktuellen Stand des Mitarbeiters im Unternehmen bieten zu können. Dazu gehören unter Anderem Auszubildende, Angestellten, Führungskräfte oder stündlich bezahlten Mitarbeiter.\footnote{Vgl. \cite{SAPSE2024a}} Diese beiden Parameter zusammen stellen die aktuelle Position eines Mitarbeiters im Unternehmen dar.5 Die Abrechnungsbereiche können verwendet werden um z.B. das Datum der Gehaltszahlung festlegen zu können. So können Mitarbeiter z.B. am 28. oder aber am 1. eines Monats ihre Gehaltszahlung erhalten.\footnote{Vgl. \cite{SAPSE2024a}} 

\subsection{Aufbauorganisation}
Die letzte und dritte Struktur die im HCM-Modul abgebildet wird ist die Aufbauorganisation, welche beschreibt wie die unterschiedlichen Organisationseinheiten, Mitarbeitern und den von ihnen eingenommenen Personen miteinander interagieren. Der Grundbaustein hierfür sind sogenannte Objekte, alle Teile der Organisation beschreiben. Alle Eigenschaften eines Objektes werden durch seine Infotypen bestimmt. Dadurch wird ein Zusammenhang zwischen der zuständigen Organisationseinheit, der Planstelle, dem Mitarbeiter und der Stelle/Kostenstelle hergestellt.\footnote{Vgl. \cite{SSSUM2019a}} Hierbei repräsentiert eine Kostenstelle eine beliebige Kostenentstehung im Unternehmen. Sie haben den Vorteil das, wenn Kostenstellen effektiv eingesetzt werden, die Entstehung von Kosten in der Abrechnung genau einer Quelle zugeordnet werden kann.\footnote{Vgl. \cite{SAPSE2024b}} Die Planstelle wird einer Organisationseinheit zugewiesen, welche eine Aufgabe in einem Bereich des Unternehmens darstellt. 

Durch die Oben genannten Komponenten einer Aufbauorganisation kann so eine Baumstruktur oder Organigramm eines Unternehmens erstellt werden. Diese beginnt mit den Organisationseinheit, welche dann eine Stelle beinhalten, welche durch Planstellen umgesetzt wird.\footnote{Vgl. \cite{SAPSE2024c}} 

\subsection{Stammdaten}
Stammdaten unterscheiden sich zu Bewegungsdaten dadurch das sie nur einmal im System gesetzt werden müssen, wonach sie referenziert werden können um z.B. weitere Daten eines Objektes lediglich referenzieren zu können als das sie neu eingetragen werden müssen.\footnote{Vgl. \cite{SSSUM2019a}} Im Modul HCM sind dies unter anderem die Personalnummer und die Referenzpersonalnummer. Diese können entweder durch das System oder manuell durch einen Mitarbeiter gesetzt werden und ermöglichen es einen Mitarbeiter eindeutig zu identifizieren.\footnote{Vgl. \cite{SSSUM2019a}} Die Referenzpersonalnummer wird dafür verwendet um eine Mehrfachbeschäftigung eines Mitarbeiters abbilden zu können. Durch das verwenden einer Referenzpersonalnummer können bestehende Personaldaten referenziert werden. 

\subsection{HCM-Prozesse}
Um ein ERP-System in einem Unternehmen einsetzen zu können müssen nicht nur Strukturen des Unternehmens abgebildet werden könne. Der wichtigere Bestandteil ist die Fähigkeit der Benutzer Prozesse innerhalb des Systems anstoßen zu können um von der nun digitalen Abbildung des Unternehmens profitieren zu können. Hierdurch können Zeit und Kosten gespart werden, da zeitintensive händisch voran getriebene Prozesse durch automatisierte Prozesse im System ersetzt werden können. In dem Modul HCM sind eine Vielzahl von Prozessen bereits enthalten welche es ermöglichen das Personal eines Unternehmens effektiv verwalten zu können. Die Prozesse im HCM-Modul umfassen ebenfalls die Funktionen die eigentlichen Strukturen aufzubauen und verändern zu können. 

\subsection{Organisationsmanagement}
Das Organisationsmanagement ist einer der zentralen Prozesse des HCMs welcher es Benutzern ermöglicht die im System abgebildete Struktur des Unternehmens fortlaufend anzupassen. Es umfasst das anlegen und ändern von Organisationseinheit, das abgrenzen und erstellen von Stellen/Planstellen und bietet die Möglichkeit einzelne Organisationseinheiten auf z.B. personelle Anforderungen zu analysieren.\footnote{Vgl. \cite{SSSUM2019a}} Durch diese Funktionen kann z.B. ein potentieller personeller Mangel erkannt werden, bevor mehrere Mitarbeiter in den Ruhestand gehen. Hier kann dann frühzeitig nach neuen Arbeitskräften im zur Verfügung stehenden Mitarbeiter-Pool gesucht werden. Ebenfalls können die vorhandenen Daten erweitert werden um Planungsszenarien zu schaffen beziehungsweise um Simulationen von möglichen Entwicklung zu erstellen. 

Für das Organisationsmanagement werden ebenfalls wieder die jeweilig relevanten Daten der Mitarbeiter und Kostenstellen verwendet. Diese Objekte und ihre Infotypen werden dafür verwendet um Veränderungen in den Strukturen des Unternehmens ebenfalls im HCM-Modul umzusetzen. 

Organisationsmanagement umfasst somit sämtliche Prozesse die für das Verwalten der Gesamten Unternehmensstruktur innerhalb des HCM-Moduls.\footnote{Vgl. \cite{SAPSE2024a}}

\subsection{Personaladministration}
Die Komponente Personaladministration stellt Funktionen zur Verwaltung der Belegschaft in einem Unternehmen zur Verfügung. Zusätzlich zu diesen Funktionen integriert sich die Komponente mit anderen Komponenten wie der Komponente Personalbeschaffung, Personalentwicklung und dem Talentmanagment.\footnote{Vgl. \cite{SAPSE2022}} Die Komponente Personaladministration ermöglicht es umfassende Datensätze zu Mitarbeitern anzulegen, welche z.B. beim füllen von Planstellen im Unternehmen zu rate gezogen werden können. Veränderungen der Position von Mitarbeitern durch z.B. Kündigungen oder Positionswechseln werden mit sogenannten Personalmaßnahmen beschrieben. Personalmaßnahmen sind Infotypen welche verändert, hinzugefügt oder gelöscht werden, wenn es personelle Veränderungen gibt.\footnote{Vgl. \cite{SAPSE2023a}} Ein weitere wichtiger Bestandteil der herkömmlichen Personaladministration besteht aus dem erstellen von Berichten für beispielsweise die höher liegenden Führungsebenen. Auch hier bietet die Komponente Personaladministration eine Vielzahl von standardmäßig verfügbaren Berichten die automatisiert erstellt werden können. Der Verwaltungsaufwand der Personaladministration kann ebenfalls durch eine Möglichkeit zum Self-Service reduziert werden. Self-Service beschreibt, im Kontext eines ERP-Systems die Möglichkeit das Mitarbeiter ihre Personal-Daten eigenständig abrufen und ändern können um diese aktuelle zu halten.\footnote{Vgl. \cite{SAPSE2023}} 

\subsection{Personalbeschaffung}
Die Personalentwicklung und -beschaffung beschäftigt sich mit dem erschließen von neuen personellen Ressourcen durch das verwalten und anwerben von neuen Personen. Ebenfalls kann die Entwicklung von Personen über diese Komponente verwalten und gesteuert werden. Dabei bietet die Komponente nicht nur Möglichkeiten zur außerbetrieblichen Anwerbung von Personen sonder auch die Auswahl von sich bereits im Unternehmen befindenden Personen.\footnote{Vgl. \cite{SSSUM2019a}} Bewerber und deren Daten werden in einer Bewerberdatenbank gespeichert. Bewerber können ebenfalls in Bewerbergruppen unterteilt werden um beispielsweise zwischen aktiven externen Bewerbern zu unterscheiden. Eine grobe Unterteilung erfolgt ebenfalls durch sogenannte Bewerberkreise, welche Bewerber nach ihrem Beschäftigungsverhältnis unterteilen.\footnote{Vgl. \cite{SSSUM2019}} Die Personalbeschaffung umfasst eine Vielzahl von Schritten welche durchlaufen werden müssen, bis eine Person bei einem Unternehmen angestellt  wird. Auch diese können im Modul HCM abgebildet werden.\footnote{Vgl. \cite{SSSUM2019a}}

\subsection{Personalentwicklung und Talentmanagement}
Nachdem Personen für ein Unternehmen angeworben wurden ist es von Interesse deren Werdegang zu dokumentieren und während ihrer Karriere in dem Unternehmen weiter zu verfolgen. Die Funktionen in der Personalentwicklungs-Komponente bringen die Möglichkeit mit sich das Qualifikations-Objekte mit einer Person verknüpft werden können. Diese Qualifikationen können in einem Qualifikations-Katalog gepflegt werden.\footnote{Vgl. \cite{SSSUM2019a}} Dieser ermöglicht eine Gruppierung der Qualifikationen nach beispielsweise Arbeitsbereichen wie Anwendungsentwicklung oder Personalverwaltung. Diese Gruppen können in der Personalentwicklungs-Komponente beliebig erweitert werden. In der Personalentwicklung fließen mehrere wichtige Funktionen zusammen. Hier können Stellen und Planstellen mit den zur Verfügung stehenden Personen abgeglichen. Der Personalvergleich ermöglicht es die Qualitäten zweier Personen miteinander zu vergleichen und die geeignetere für die Planstelle auszuwählen.\footnote{Vgl. \cite{SSSUM2019a}} 

Besitzen die vorhandenen Personen nicht die notwendigen Qualifikationen kann es von Sinn seien neue Karriereziele für besagte Personen abzustecken. Diese Funktionen werden durch die Talentmanagement-Komponente geboten.\footnote{Vgl. \cite{SSSUM2019}} Bei der Laufbahnplanung werden vorhandene Qualifikationen der Person betrachtet um neue Qualifikationen planen zu können. So kann die Entwicklung des Personals gesteuert und so gelenkt werden das sie auf lange Sicht dem Wachstum und Fortschritt des Unternehmens hilft. 

\subsection{Performance-Management}
Als Führungskraft und Manager kann es von großer Bedeutung seien, wenn die Möglichkeit vorhanden ist die Effizienz und Arbeitskraft des eigenen Teams bewerten zu können. Das integrierte Performance Managment des HCM-Moduls bietet eine vielzahl von Benutzerfreundlichen Oberflächen durch welche ein eigener Performance-Managment-Prozess definiert werden kann.\footnote{Vgl. \cite{SAPSE2024d}} So können eigene Werte und Ziele für das Unternehmen und dessen Mitarbeitern festgelegt werden. Diese Ziele können auch für nur spezifische Personengruppen, wie z.B. nur einem Team, zugeordnet werden. Ebenfalls in der Komponenten enthalten sind Masken welche direkt von einer Person oder einem Mitarbeiter verwendet werden können um eine eigenständige Performance-Review durchzuführen. Somit ist eine Möglichkeit zum Self-Service gegeben. Nach einer Bewertung können Manager, basierend auf dem ihnen zur Verfügung stehenden Beurteilungsdokument dem Mitarbeiter bestimmte Weiterbildungs- oder Schulungsmaßnahmen vorschlagen.\footnote{Vgl. \cite{SAPSE2024d}} 

\newpage